\documentclass[11pt,a4paper]{article}

\usepackage{style2017}
\newcounter{numexo}
\setcellgapes{1pt}

\begin{document}


\begin{NSI}
{Exercice}{Fonction en Python}
\end{NSI}


\addtocounter{numexo}{1}
\subsection*{\Large Exercice \thenumexo }
On donne le script python suivant.
\begin{center}
\includegraphics[scale=0.8]{img/pointe.eps}
\end{center}
\begin{enumerate}
\item Que renvoie l'appel de la fonction \textbf{ligne(10,3)}?
\item On exécute le script. Quel est l'affichage obtenu ?
\end{enumerate}


\addtocounter{numexo}{1}
\subsection*{\Large Exercice \thenumexo }
\begin{enumerate}
\item Créer une fonction "somme\_carre" qui prend en paramètre deux nombres a et b et renvoie la somme des carrés des deux nombres.
\item Créer une fonction "double\_produit" qui prend en paramètre deux nombres a et b et renvoie le double du produit de ces deux nombres.
\item On veut créer une fonction id\_rem qui prend en paramètre deux nombres a et b et renvoie le carré de la somme de ces deux nombres.
\begin{enumerate}
\item Écrire cette fonction.
\item Écrire cette fonction en utilisant les fonctions somme\_carre et double\_produit.
\end{enumerate}
\end{enumerate}

\addtocounter{numexo}{1}
\subsection*{\Large Exercice \thenumexo }
\begin{enumerate}
\item Créer une fonction "somme\_arithm" qui prend en paramètre un nombre entier N et retourne la somme de tous les nombres entiers inférieurs ou égaux à N.
\item Créer une fonction "somme\_carre" qui prend en paramètre un nombre entier N et retourne la somme des carrés des nombres entiers inférieurs ou égaux à N.
\item Créer une fonction "somme\_inverse" qui prend en paramètre un nombre entier N et retourne la somme des inverses des nombres entiers inférieurs ou égaux à N.
\item Créer une fonction "somme\_inverse\_carre" qui prend en paramètre un nombre entier N et retourne la somme des carrés des inverses des nombres entiers inférieurs ou égaux à N.
\end{enumerate}

\newpage
\addtocounter{numexo}{1}
\subsection*{\Large Exercice \thenumexo }
\begin{enumerate}
\item Écrire une fonction "zen" qui renvoie la ligne de 10 caractères : ZNZNZNZNZN
\item Modifier cette fonction pour prendre en paramètre le nombre n de caractères.
\begin{itemize}
\item si n est pair, la ligne commence par Z et se finit par N;
\item si n est impair, la ligne comment et finit par la lettre Z.
\end{itemize}
\item Modifier cette fonction pour prendre aussi en paramètre chaque caractère, N et Z étant les valeurs par défaut de la fonction.
\item Écrire un script avec la fonction zen pour reproduire la figure ci-dessous.
\begin{center}
\includegraphics[scale=0.8]{img/zen_lignes.eps}
\end{center}
\item Écrire une fonction "pluszen" qui utilise la fonction zen pour renvoyer p lignes et n colonnes de 2 caractères décalés à chaque ligne. Par exemple, le code ci-dessous:
\begin{center}
\includegraphics[scale=0.8]{img/pluszen.eps}
\end{center}
renvoie l'affichage:
\begin{center}
\includegraphics[scale=0.8]{img/pluszen-fig.eps}
\end{center}
\end{enumerate}

\addtocounter{numexo}{1}
\subsection*{\Large Exercice \thenumexo }
\begin{enumerate}
\item Créer une fonction "est\_impair" qui prend en paramètre un nombre entier et renvoie vrai si le nombre est impair et faux s'il ne l'est pas.
\item Créer une fonction "est\_multiple" qui prend en paramètre 2 nombres entiers et qui renvoie vrai si le premier nombre est un multiple du second et faux sinon.
\item Un nombre entier est dit \textbf{premier} s'il est divisible seulement par 1 et lui-même. Écrire une fonction "est\_premier" qui prend en paramètre un nombre entier et renvoie vrai s'il est premier et faux sinon.
\end{enumerate}

%\addtocounter{numexo}{1}
%\subsection*{\Large Exercice \thenumexo }
%On dit que deux nombres entiers A et B sont amicaux si la somme des diviseurs de A est égale à B et en même temps la somme des diviseurs de B est égale à A. Attention, on ne prend pas en compte A et B comme diviseur.
%\begin{enumerate}
%\item Créer une fonction "somme\_diviseur" qui prend en paramètre un nombre entier N et retourne la somme des diviseurs de N (on ne doit pas prendre  en compte N comme diviseur).
%\item Créer une fonction "nombres\_amicaux" qui prend en paramètre 2 nombres entiers A et B et retourne True s'ils sont amicaux et False sinon.
%\item Vérifier que les nombres 220 et 284 sont amicaux. De même avec 12 285 et 14 595.
%\end{enumerate}


\addtocounter{numexo}{1}
\subsection*{\Large Exercice \thenumexo }
Un palindrome est un mot qui peut se lire dans les deux sens, comme par exemple LAVAL et NON.
\begin{enumerate}
\item Créer une fonction renverse qui prend en paramètre une chaine de caractères et renvoie cette même chaine écrite dans l'autre sens.
\item Écrire une fonction palindrome qui prend en paramètre une chaine de caractères et renvoie vrai si cette chaine est un palindrome.
\item Faire de même avec les nombres entiers en écriture décimale.
\end{enumerate}


\end{document}

